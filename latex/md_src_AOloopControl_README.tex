This module allows control of adaptive optics (A\+O) control loops.

Source code in \hyperlink{AOloopControl_8c}{A\+Oloop\+Control.\+c} and \hyperlink{AOloopControl_8h}{A\+Oloop\+Control.\+h}

Main features\+:
\begin{DoxyItemize}
\item uses shared memory to communicate between processes and link to wavefront sensor camera/data and deformable mirrors
\item multiple loops can run simultaneously, independantly or together
\item single process can control multiple loops
\item Can use either C\+P\+U(s) or G\+P\+U(s) for computations
\end{DoxyItemize}

\section*{Top-\/level script\+: aolrun}

The top-\/level script is aolrun.

Script aolrun initializes the control loop by reading configuration input parameters, and then responds to changes in the \char`\"{}wavefront sensor input\char`\"{} image. Control loops are numbered. The loop number is read from local file L\+O\+O\+P\+N\+U\+M\+B\+E\+R.

\subsection*{Required input for aolrun, prior setup}

The following table lists all input / files required prior to running aolrun

\begin{TabularC}{3}
\hline
\rowcolor{lightgray}{\bf Variable Name }&{\bf Description }&{\bf Setting location  }\\\cline{1-3}
L\+O\+O\+P\+N\+U\+M\+B\+E\+R &loop number \mbox{[}integer\mbox{]} &local file L\+O\+O\+P\+N\+U\+M\+B\+E\+R \\\cline{1-3}
A\+Oconf\mbox{[}loop\mbox{]}.D\+Mname &shared memory D\+M image (for correction) &image aol$<$\+L\+O\+O\+P\+N\+U\+M\+B\+E\+R$>$\+\_\+dm\+C \\\cline{1-3}
A\+Oconf\mbox{[}loop\mbox{]}.D\+Mname\+R\+M &shared memory D\+M image (for response matrix acq) &image aol$<$\+L\+O\+O\+P\+N\+U\+M\+B\+E\+R$>$\+\_\+dm\+R\+M \\\cline{1-3}
A\+Oconf\mbox{[}loop\mbox{]}.W\+F\+Sname &shared memory W\+F\+S image &image aol$<$\+L\+O\+O\+P\+N\+U\+M\+B\+E\+R$>$\+\_\+wfs \\\cline{1-3}
A\+Oconf\mbox{[}loop\mbox{]}.D\+M\+M\+O\+D\+E\+Sname &D\+M modes &image aol$<$\+L\+O\+O\+P\+N\+U\+M\+B\+E\+R$>$\+\_\+\+D\+Mmodes \\\cline{1-3}
A\+Oconf\mbox{[}loop\mbox{]}.resp\+Mname &response matrix &image aol$<$\+L\+O\+O\+P\+N\+U\+M\+B\+E\+R$>$\+\_\+\+Resp\+M \\\cline{1-3}
A\+Oconf\mbox{[}loop\mbox{]}.contr\+Mname &control matrix &image aol$<$\+L\+O\+O\+P\+N\+U\+M\+B\+E\+R$>$\+\_\+\+Contr\+M \\\cline{1-3}
A\+Oconf\mbox{[}loop\mbox{]}.name &loop name &local file ./conf/conf\+\_\+\+L\+O\+O\+P\+N\+A\+M\+E.txt \\\cline{1-3}
A\+Oconf\mbox{[}loop\mbox{]}.G\+P\+U &number of G\+P\+Us used &local file ./conf/conf\+\_\+\+G\+P\+U.txt \\\cline{1-3}
A\+Oconf\mbox{[}loop\mbox{]}.G\+P\+Uall &skip C\+P\+U image scaling and go straight to C\+P\+U ? &local file ./conf/conf\+\_\+\+G\+P\+Uall.txt \\\cline{1-3}
A\+Oconf\mbox{[}loop\mbox{]}.A\+O\+L\+C\+O\+M\+P\+U\+T\+E\+\_\+\+T\+O\+T\+A\+L\+\_\+\+A\+S\+Y\+N\+C &compute image total in separate thread ? &local file ./conf/conf\+\_\+\+C\+O\+M\+P\+U\+T\+E\+\_\+\+T\+O\+T\+A\+L\+\_\+\+A\+S\+Y\+N\+C.txt \\\cline{1-3}
M\+A\+T\+R\+I\+X\+\_\+\+C\+O\+M\+P\+U\+T\+A\+T\+I\+O\+N\+\_\+\+M\+O\+D\+E &use single combined act-\/wfs matrix ? &local file ./conf/conf\+\_\+\+C\+Mmode.txt \\\cline{1-3}
\end{TabularC}


\section*{Step-\/by-\/step instruction for typical operation}

Control loops are numbered -\/ this is how multiple loops can work together. $<$\+L\+O\+O\+P\+N\+B$>$ can take any positive integer value (0, 1, 2, etc...)

Important notes\+:
\begin{DoxyItemize}
\item $<$executable$>$ is the name given to the executable program
\item some commands below are stand-\/alone executable scripts, others are commands to be issued within the main execuatable, which are shown starting with \char`\"{}$>$ \char`\"{}.
\item stand alone scripts will need to be created if they do not exist\+: the scripts are provided in this page.
\end{DoxyItemize}

\subsubsection*{Directories, configurations}

System configurations are saved in independent directories named ./confxxx/.

A configuration consists of\+:
\begin{DoxyItemize}
\item fmodes.\+fits \+: Modes definition
\item modesfreqcpa.\+fits \+: Modes spatial frequency (C\+P\+A)
\item refwfs.\+fits \+: W\+F\+S reference
\item respm.\+fits \+: Response matrix
\item cmat.\+fits \+: Control matrix
\item additional control matrixes, named\+: cmat\+\_\+$<$beta$>$\+\_\+$<$nb\+Mrm$>$.\+fits
\begin{DoxyItemize}
\item $<$beta$>$ is the explonent for low order modes enhancement in the S\+V\+D\+: for example 0.\+43
\item $<$nb\+Mrm$>$ is the number of modes removed (low eigenvalues)
\end{DoxyItemize}
\item eigenmodes\+M\+\_\+$<$beta$>$.\+fits \+: eigenmodes
\item eigenmodesresp\+M\+\_\+$<$beta$>$.\+fits \+: W\+F\+S response to eigenmodes
\item T\+Pmask.\+fits \+: Telescope pupil mask (optional)
\item T\+Pind.\+fits \+: modes to be removed from control (optional)
\end{DoxyItemize}

Scripts are used to load and save configurations as follows\+: \begin{DoxyVerb}./saveconf xxx
./loadconf xxx
\end{DoxyVerb}


saveconf source\+:

\begin{DoxyVerb}mkdir -p conf$1
cp fmodes.fits ./conf$1/
cp refwfs.fits ./conf$1/
cp respm.fits ./conf$1/
cp cmat.fits ./conf$1/
cp cmat_*_*.fits ./conf$1/
cp AOloop*.conf ./conf$1/
\end{DoxyVerb}


loadconf source\+:

\begin{DoxyVerb}cp ./conf$1/fmodes.fits .
cp ./conf$1/refwfs.fits .
cp ./conf$1/respm.fits .
cp ./conf$1/cmat.fits .
cp ./conf$1/AOloop*.conf .
\end{DoxyVerb}


\subsubsection*{O\+P\+T\+I\+O\+N\+A\+L V\+A\+R\+I\+A\+B\+L\+E\+S}

The following variables are optional, and may be specified as command line variable\+:
\begin{DoxyItemize}
\item A\+O\+L\+C\+A\+M\+D\+A\+R\+K\+: dark level for the camera images. By default, it is assumed that the camera images fed to the software are dark-\/subtracted. If they are not, it is possible to offset them by a dark value by entering\+:

A\+O\+L\+C\+A\+M\+D\+A\+R\+K=100.\+0
\end{DoxyItemize}

It is best to keep optional variable in the C\+L\+I startup script, which is automatically executed by the program upon startup. By default, the file name is 'C\+L\+Istartup.\+txt'. It can also be changed and loaded by using the '-\/s' option when launching the executable. Example startup script\+: \begin{DoxyVerb}aolnb 1
AOLCAMDARK=100.0
\end{DoxyVerb}






\subsection*{C\+O\+N\+F\+I\+G\+U\+R\+A\+T\+I\+O\+N}

\subsubsection*{(1) C\+R\+E\+A\+T\+E/\+E\+D\+I\+T A\+Oloop$<$\+L\+O\+O\+P\+N\+B$>$.\+conf}

The configuration for running the A\+O loop is in this file. Shared memory images names are specified, as well as loop configuration paramaters.

The function \hyperlink{AOloopControl_8h_a5c3b0bdcff0ed83ed3f9f217b14c014d}{A\+Oloop\+Control\+\_\+make\+Template\+A\+Oloopconf(long loopnb)} creates a template that can be edited\+: \begin{DoxyVerb}<executable>
> aolmkconf <LOOPNB>
\end{DoxyVerb}


\subsubsection*{(2) C\+R\+E\+A\+T\+E M\+O\+D\+E\+S}

If possible, create the T\+Pmask.\+fits file which contains the D\+M illumination. Illuminated actuators are set to 1, others to 0.

For purely Fourier Modes\+: \begin{DoxyVerb}./mkModes <confindex> <CPAmax>
\end{DoxyVerb}


confindex\+: index for the configuration, for example, 002 will save modes in conf002 directory C\+P\+Amax\+: maximum spatial frequency in the basis of modes, in unit of Cycle Per Aperture.

The mode file name is ./conf$<$confindex$>$/fmodes.fits.

This script calls the function A\+Oloop\+Control\+\_\+mk\+Modes(char $\ast$\+I\+D\+\_\+name, long msize, float C\+P\+Amax, float delta\+C\+P\+A, double xc, double yx, double r0, double r1) in \hyperlink{AOloopControl_8c}{A\+Oloop\+Control.\+c}.

Optional input\+:
\begin{DoxyItemize}
\item T\+Pmask.\+fits \+: telescope pupil mask. If this file is present in the configuration directory, it will be used as a multiplicative mask (with convolotion to avoid sharp transitions) to the modes
\item T\+Pind.\+fits \+: modes to be excluded from the control
\end{DoxyItemize}

mk\+Modes source\+:

\begin{DoxyVerb}# args: <confindex> <maxCPA>
EXPECTED_ARGS=2

if [ $# -ne $EXPECTED_ARGS ]
then
  echo
  echo "-------- create control modes ------------"
  echo "Usage:  $0 <configuration index> <maxCPA>"
  echo "   INPUT:  <configuration index>  : index, 001, 002, 003 etc.."
  echo "   INPUT:  <max CPA>              : maximum spatial frequ [cycles per aperture]"
  echo " EXAMPLE: $0 002 4.0"
  echo
  exit
fi





mkdir -p conf$1

echo "MODES_MAX_CPA      $2" > ./conf$1/fmodes.conf.txt


Cfits << EOF
loadfits "./conf$1/TPmask.fits" Mmask
loadfits "./conf$1/TPind.fits" emodes
aolmkmodes fmodes 50 $2 0.8 24.5 24.5 21.0 6.5
savefits fmodes "!./conf$1/fmodes.fits"
savefits modesfreqcpa "!./conf$1/modesfreqcpa.fits"
exit
EOF
\end{DoxyVerb}


\subsubsection*{(3) F\+I\+N\+D\+I\+N\+G L\+O\+O\+P L\+A\+T\+E\+N\+C\+Y (O\+P\+T\+I\+O\+N\+A\+L)}

\begin{DoxyVerb}./MeasureTD <confindex>
\end{DoxyVerb}


This script will create the \char`\"{}\+Time\+Delay\+R\+M.\+txt\char`\"{}, which gives signal strength as a function of frame delay.

The first peak of this curve indicates the time delay and should be updated in the acqu\+Resp\+M script below.

Measure\+T\+D source\+: \begin{DoxyVerb}cp ./conf$1/fmodes.fits .
<executable> << EOF
aolnb <LOOPNB>
aolacqresp 20 0.05 10 -1 1
exit
EOF
\end{DoxyVerb}


Once the delay is known, edit the acqu\+Resp\+M script (see below) accordingly.

\subsubsection*{(4) A\+C\+Q\+U\+I\+R\+E R\+E\+S\+P\+O\+N\+S\+E M\+A\+T\+R\+I\+X}

\begin{DoxyVerb}./acquRespM 000
\end{DoxyVerb}


Edit script to change modulation amplitude, number of images per D\+M state, etc...

This script calls function \hyperlink{AOloopControl_8h_ab51676f77cfd9e45abd24cb0a68a08e7}{Measure\+\_\+\+Resp\+\_\+\+Matrix(long loop, long Nb\+Ave, float amp, long nbloop, long f\+Delay, long N\+Biter)} in module \hyperlink{AOloopControl_8c}{A\+Oloop\+Control.\+c}.

The response matrix acquision will go for a large number of iterations, and data is progressively averaged to improve S\+N\+R. The last parameter of the acqu\+Resp\+M command in the script is the number of matrixes averaged. You can either wait for all iterations to complete or C\+T\+R\+L-\/\+C the process. If you C\+T\+R\+L-\/\+C the process, load the result in the configuration directory\+: \begin{DoxyVerb}cp respm.fits ./conf<xxx>/
cp refwfs.fits ./conf<xxx>/
\end{DoxyVerb}


output\+: respm.\+fits, refwfs.\+fits

acqu\+Resp\+M source\+: \begin{DoxyVerb}# Acquires Response Matrix
# Parameters for aolacqresp are:
# number of images per DM state
# modulation amplitude [um DM displacement]
# number of loop per matrix generation
# frame delay offset [number of frames]
# number of matrixes generated (will be continuously averaged)

# creates files "respm.fits" and "refwfs.fits"

./loadconf $1
<executable> -n aolacqresp<< EOF
aolnb 0
aolacqresp 10 0.02 2 2 100
exit
EOF
cp respm.fits ./conf$1/
cp refwfs.fits ./conf$1/
\end{DoxyVerb}


\subsubsection*{(5) C\+O\+M\+P\+U\+T\+E C\+O\+N\+T\+R\+O\+L M\+A\+T\+R\+I\+X\+E\+S}

\begin{DoxyVerb}./cmmake <confindex> <beta> <#modes removed>
\end{DoxyVerb}


output\+: cmat\+\_\+$<$beta$>$\+\_\+$<$\#modes removed$>$.fits

Also writes eigenvalues in file \char`\"{}eigenv.\+dat\char`\"{}. Check that number of modes removed is consistent with eigenvalues.

$<$beta$>$ is the enhancement factor for low order aberrations. For beta=0, there is no enhancement (straight S\+V\+D), while beta $>$0 forces the lowest modes to be low-\/order. Typical values from 0.\+50 to 4.\+00.

cmmake source\+: \begin{DoxyVerb}#
# Make control matrix
# arg 1: configuration directory
# arg 2: Beta coefficient (typically 0.50 to 4.00), enhances low order aberrations for SVD
# arg 3: max number of modes removed

cp ./conf$1/modesfreqcpa.fits .
cp ./conf$1/fmodes.fits .
cp ./conf$1/respm.fits .

<executable> << EOF
aolnb 0
loadfits respm.fits respm
loadfits fmodes.fits modesM
aolcmmake $3 respm cmat $2
savefits evecM "!evecM.fits"
exit
EOF
mv cmat_*_*.fits ./conf$1/
mv evecM.fits ./conf$1/
mv eigenv.dat ./conf$1/
mv eigenmodesM*.fits ./conf$1/
mv eigenmodesrespM*.fits ./conf$1/
\end{DoxyVerb}


The script calls compute\+\_\+\+Control\+Matrix(long loop, long N\+B\+\_\+\+M\+O\+D\+E\+\_\+\+R\+E\+M\+O\+V\+E\+D, char $\ast$\+I\+D\+\_\+\+Rmatrix\+\_\+name, char $\ast$\+I\+D\+\_\+\+Cmatrix\+\_\+name, char $\ast$\+I\+D\+\_\+\+V\+Tmatrix\+\_\+name, double Beta) in module \hyperlink{AOloopControl_8c}{A\+Oloop\+Control.\+c}.





\subsection*{R\+U\+N\+N\+I\+N\+G T\+H\+E L\+O\+O\+P}

\subsubsection*{(6) S\+E\+T U\+P / L\+O\+A\+D C\+O\+N\+F\+I\+G\+U\+R\+A\+T\+I\+O\+N}

Steps above can be repeated to create several configurations. When adopting a configuration\+:
\begin{DoxyItemize}
\item Choose a control matrix cp conf000/cmat\+\_\+1.\+00\+\_\+003.\+fits conf000/cmat.\+fits
\item load configuration ./loadconf 000
\end{DoxyItemize}

\subsubsection*{(7) R\+U\+N L\+O\+O\+P}

Start T\+W\+O instances of the program

In process \#1, type \+: \begin{DoxyVerb}> aolnb <LOOPNB>
> aolrun
\end{DoxyVerb}


In process \#2, type \+: \begin{DoxyVerb}> aolnb <LOOPNB>
> aolsetgain <loopgain>
> aolon
\end{DoxyVerb}


Other useful commands \+:
\begin{DoxyItemize}
\item aolsetmaxlim $<$maximum\+\_\+value\+\_\+for\+\_\+each\+\_\+mode$>$
\item aolsetmultfb $<$block\+\_\+number$>$ $<$multiplicative\+\_\+factor$>$
\end{DoxyItemize}

\subsubsection*{(8) L\+A\+U\+N\+C\+H M\+O\+N\+I\+T\+O\+R}

Start new process, type\+: \begin{DoxyVerb}> aolmon <frequ[Hz]> <number_of_columns>
\end{DoxyVerb}


For each mode, the following info is displayed\+: \begin{DoxyVerb}[<gain_factor> <limit_factor> <multiplier_factor>]   <current_DM_value> <last_WFS_estimate> <average_val> <RMS> 
\end{DoxyVerb}
 