Adaptive Optics Wavefront control tools for high contrast imaging

Uses shared memory for fast low-\/latency communication to hardware

Includes a simulator of hardware (D\+M, Camera)\+: the same code can be tested on actual hardware or in a simulation mode.

Supports multiple simultaneously running control loops and data logging.

\subsection*{Downloading source code}

Latest distribution is on \href{https://github.com/oguyon/AdaptiveOpticsControl}{\tt github}. You can clone the repository, or download the latest .tar.\+gz distribution.

\subsection*{Compilation}

The source code follows the standard G\+N\+U build process\+:

./configure

make

make install

\subsection*{Documentation}

Please consult the \href{http://oguyon.github.io/AdaptiveOpticsControl/}{\tt online documentation}.

\subsection*{Libraries}

The following libraries are used\+:
\begin{DoxyItemize}
\item readline, for reading the command line input
\item flex, for parsing the command line input
\item bison, to interpret the command line input
\item fftw, for performing Fourier Transforms
\item gsl, for math functions and tools
\item fitsio, for reading and writing F\+I\+T\+S image files
\end{DoxyItemize}

\subsection*{Source Code Architecture}

Written in C. The main is a command line interface (C\+L\+I). Source code is in \hyperlink{CLIcore_8c}{C\+L\+Icore.\+c} and \hyperlink{CLIcore_8h}{C\+L\+Icore.\+h}. Key data structures (such as the image data structure) are declared in \hyperlink{CLIcore_8h}{C\+L\+Icore.\+h}.

\subsection*{How to run the turbulence simulator}

Copy scripts from ./src/\+Atmospheric\+Turbulence/scripts directory to working directory. Edit and execute the main script \char`\"{}runturb\char`\"{}

\subsection*{Credits}

This software was developed with support from the National Science Foundation (award \#1006063), N\+A\+S\+A and N\+A\+O\+J/\+Subaru Telescope. 